\documentclass[12pt]{jreport}
%\usepackage[dvips]{graphicx} % 図の貼り付け用
\usepackage[dvipdfmx]{graphicx}
\usepackage{ics} % ICS卒論・修論スタイルファイル
\usepackage{makeidx} %索引生成用パッケージ
%\usepackage{tabularx}% 横幅指定で表を作成
%\usepackage{latexsym} % 数学記号用パッケージ
\usepackage{hhline}
\usepackage{here}

\usepackage{url}
\urlstyle{same}


\newtheorem{definition}{定義}[chapter]
\newtheorem{algorithm}{アルゴリズム}[chapter]

%% end of local definitions

\def\epsfsize#1#2{\ifnum#1>\hsize\hsize\else#1\fi}

\begin{document}

\papercode{ICS-24B-065}
\title{深層学習を用いた2D-LiDARによるruru歩行者追跡の高度化}
\affiliation{工学部情報工学科}
%\affiliation{理工学研究科 情報システム工学専攻}
%\affiliation{理工学研究科 数理電子情報系専攻\\情報システム工学コース}
\author{廣中 優平}
\date{令和6年2月8日提出}
\supervisor{小林 貴訓教授}
\labname{小林研究室}
\studentID{20TI065}
\maketitle



\setcounter{page}{1}
\chapter*{概要}
 \pagenumbering{roman} % 消さない
 \addcontentsline{toc}{chapter}{概要} %消さない
人の動きを追跡,分析することは,様々な場面で非常に重要になっている.近年では公共空間で人と共存するロボットも多く登場し,人の動きを正しく追跡することの重要性は一層増している.
 
 現在行われている歩行者追跡の多くは,カメラ画像を用いて行われている.カメラ画像は情報量が多く,正確な計測を行うことができる.一方で,公共空間での撮影においては,不特定多数の人物の顔などが記録されるため,プライバシーの問題が発生する.そこで本研究では,足元の高さに設置した2D-LiDARで取得できるデータを使った歩行者追跡を提案する.

 以前より小林研究室では,足元の高さに設置した2D-LiDARを用いた歩行者追跡が行われてきた.先行研究においては,クラスタリングを用いた人物検出とカルマンフィルタによる追跡を用いて,精度の高い歩行者追跡が実現されている.一方で,複数人が歩行する場合に遮蔽が多発すると追跡性能が落ちる点,処理速度が遅いためにリアルタイムでの追跡には向かない点などが問題点として挙げられていた.
 

本研究では,カメラ画像に対する複数物体追跡の技術を2D-LiDAR画像に応用することで,前述の問題点の解消を目指す.複数物体追跡技術の中で,特に優れた性能を持っているトラッカーとしてByteTrackがある.今回は,ByteTrackのアルゴリズムを導入することで,足元の高さに設置した2D-LiDARを使った歩行者追跡の高度化を目指した.



 本稿では,本研究の手法と先行研究の手法を同じ歩行に対して適用することで,追跡性能の比較を行った.追跡の持続性という観点では,本研究の手法の有用性を確認できた,一方で,歩行者同士が接近する場合に追跡精度が低下する点など,いくつかの課題も確認された.
 


%%% 謝辞 %%%

\chapter*{謝辞}
\addcontentsline{toc}{chapter}{謝辞} % 消さない


研究ならびに生活面においてご指導を賜りました小林教授,鈴木助教に深く感謝致します.

また,先輩としていつもよきアドバイスをくださり,快くデータを提供してくださりました須田悠介氏をはじめとする研究室の皆様,そして同期学生の皆様,並びに私を暖かく見守って頂いた両親はじめとする周囲のすべての皆様に深く感謝致します.

\tableofcontents % 目次生成
\listoffigures % 図目次生成
\addcontentsline{toc}{chapter}{図目次}
\listoftables % 表目次生成
\addcontentsline{toc}{chapter}{表目次}


\chapter{はじめに}
\pagenumbering{arabic} %消さないこと
\setcounter{page}{1}   %消さないこと

 \section{背景}
 公共空間において,人の流れを追跡,分析することは非常に重要である.コンビニエンスストアなどの小売店では,顧客の動きを解析することで効果的な店内配置を考えることができる.
 
 
 近年では公共空間で人と共存するロボットも多く登場し,人の流れを正しく追跡することの重要性がさらに増している.固定されたカメラやセンサーによる人流の追跡だけでなく,移動するロボットに搭載されたカメラやセンサーによる人物の検出,追跡も重要な研究テーマである.小林研究室においても,人物に自動追従する自律移動ロボットの研究\cite{sasaki}が行われており,頑健な人物追跡システムが求められている.

歩行者などの複数の物体を追跡する技術は,Multi Object Tracking (MOT)と呼ばれる.多くのMOT技術の基礎となっているSORT\cite{sort}をはじめとして,様々な手法が研究・開発されている.それらの技術は,主にカメラ画像を追跡の対象としている.カメラ画像は情報量が多いため,比較的正確な追跡を行うことができる.一方で,公共空間での撮影ではプライバシーの問題が発生する.不特定多数が利用する公共空間において,カメラ画像による計測を行うと,顔などの重要度の高い個人情報を収集することなる.カメラ画像を用いて歩行者計測を行う場合は,個人が特定されない形での情報処理や利用用途の正しい告知などが求められ,計測を行うハードルは高い.

\newpage
 \section{目的}
 本研究では,足元の高さに設置した2D-LiDARで取得できるデータを使った歩行者追跡を提案する.2D-LiDARで取得するのは足元の距離データのみであり,カメラ画像と比べると情報量が極めて少ない.プライバシーに関して考慮すべき事項が減ることで,様々な場面で歩行者追跡が気軽に行えるようになると考える.

 また,小林研究室で研究が進んでいる自律移動ロボットに搭載して,周囲の人物を追跡するシステムとしての活用も考えられる.現在は,2D-LiDARを用いて人物の肩付近を検出して追跡を行っている.複数人が周囲にいる場合での追跡精度などに課題があり,足元のデータを用いた追跡が実現されれば精度の向上につながると考えられる.


 \section{本論文の構成}
 本論文の構成は以下のようになっている.


 
    第1章 はじめに

    第2章 先行研究
     
    第3章 ByteTrack

    第4章 システムの概要

    第5章 実験

    第6章 考察

    第7章 おわりに

    第1章では,本研究の背景と目的を述べた.第2章では,これまでに行われてきた2D-LiDARによる歩行者追跡の概要について述べる.第3章では,本研究で採用した追跡アルゴリズムについて説明する.第4章では,本稿で紹介するシステムの概要を述べる.第5章では,実験の内容と結果を述べる.第6章では,実験の結果を踏まえて考察を行う.第7章では,本稿のまとめと今後の課題について述べる.


 


\chapter{先行研究}
小林研究室では,以前より足元の高さに設置した2D-LiDARを用いた歩行者追跡が行われている\cite{hasan2021person}\cite{h_b}\cite{h_m}.先行研究では,クラスタリングを用いた検出とカルマンフィルタを用いた追跡によって,歩行者追跡を実現していた.本章では先行研究の概要について説明する.
 \section{先行研究の手法}

 \subsection{クラスタリングによる検出}
先行研究\cite{hasan2021person}\cite{h_b}\cite{h_m}では,クラスタリングを用いて人物の検出を行っていた.クラスタリングアルゴリズムは,DBSCAN\cite{db}をベースとしている.DBSCANはデータの密度を基にクラスタリングを行うアルゴリズムである.



 
 2D-LiDARによって取得された人物の足は,図\ref{fig:dis}のように描画される(本稿ではこれを距離画像と呼ぶ).距離画像に対してクラスタリングを行うと,正しく人を検出することができない場合がある.右足と左足が離れている場合に2人であると検出してしまう問題や,複数人が近接する場合に1人であると検出してしまう問題が発生する.そこで先行研究では,過去フレームの画像を重ね合わせた画像(図\ref{fig:mhi})(本稿ではこれを時系列画像と呼ぶ)をクラスタリングに用いている.時系列画像は現在の足の位置だけでなく過去の動きを表現できるため,前述の問題を解消できる.




\begin{figure}[H]
   \centering
   \begin{minipage}{0.4\columnwidth}
      \centering
      \includegraphics[width=\columnwidth]{dis.png}
      \caption[距離画像]{距離画像(\cite{h_m}図3.4より)}
      \label{fig:dis}
   \end{minipage}
   \hspace{0.04\columnwidth} %
   \begin{minipage}{0.4\columnwidth}
      \centering
      \includegraphics[width=\columnwidth]{mhi.png}
      \caption[時系列画像]{時系列画像(\cite{h_m}図3.4より)}
      \label{fig:mhi}
   \end{minipage}
\end{figure}


\subsection{カルマンフィルタによる追跡}

 クラスタリングによって検出された人物に対して,重心位置を計算してカルマンフィルタを適用する(図\ref{fig:gra},図\ref{fig:kal}).カルマンフィルタでは,1ステップ前の状態から推定した位置と検出した位置より,現在の推定位置の決定と1ステップ先の推定が行われる.これによって,時間とともに動いていく人物を追跡する.

 以上のような手法によって,2D-LiDARの周辺の歩行者の追跡を行っていた.多くの場合において,複数の人物を正しく追跡することができていた.
\begin{figure}[H]
   \centering
   \begin{minipage}{0.4\columnwidth}
      \centering
      \includegraphics[width=\columnwidth]{gra.png}
      \caption[重心位置を計算]{重心位置を計算\newline 
      (\cite{h_m}図3.7より)}
      \label{fig:gra}
   \end{minipage}
   \hspace{0.04\columnwidth} 
   \begin{minipage}{0.4\columnwidth}
      \centering
      \includegraphics[width=\columnwidth]{kal.png}
      \caption[カルマンフィルタを適用]{カルマンフィルタを適用\newline 
      (\cite{h_m}図3.8より)}
      \label{fig:kal}
   \end{minipage}
\end{figure}

\newpage
 \section{先行研究の問題点}
 一方で,先行研究\cite{hasan2021person}\cite{h_b}\cite{h_m}にはいくつかの問題点も存在した.まず,追跡対象が他の人物の影に入って計測されなくなる場合に,追跡をうまく継続できないという課題があった.複数人が歩行するような環境では,遮蔽が発生した場合でも追跡を継続できるようなシステムが求められる.


\begin{figure}[H]
   \centering
   \begin{center}
      \centering
      \includegraphics[width=453pt]{occ_kal.png}
      \caption{遮蔽によって追跡がうまくいかない様子}
      \label{fig:occ_problem}
   \end{center} 
\end{figure}


 処理速度も課題となっていた.計測対象が少人数の場合は約30fps程度で追跡できるものの,人数が増えるごとに速度が低下し,7人を計測する際は10-15fpsとなってしまう.時系列画像作成,クラスタリング,カルマンフィルタの適用を並列で行うことで高速化を目指したものの,クラスタリングがボトルネックとなり高速化は実現できなかった.

また,時系列画像を作成するため,2D-LiDARが動く環境での追跡には向いていない.時系列画像は過去フレームの画像を重ね合わせて作成するため,2D-LiDAR自身が動くと正しく時系列画像を作成することはできない.これは,自律移動ロボットに搭載して人物追跡を行う事を考えると解決したい課題である.


\chapter{ByteTrack}
カメラ画像に対する複数物体追跡は,Multi Object Tracking(MOT)と総称され,様々な手法が提案されている.本研究では,その中でも極めて高い性能を発揮することで知られるByteTrack\cite{ByteTrack}のアルゴリズムを,2D-LiDAR画像に適用することを考えた.本章ではByteTrackを紹介する.

\section{Multi Object Tracking}
MOT技術とは複数物体追跡技術の総称である.MOT技術の多くは,検出(Detection)と追跡(Tracking)を切り離して考える,Tracking By Detectionという考え方を用いている.Trackingは,あるフレームで検出された物体と次のフレームで検出された物体同士をIDで紐づけることの繰り返しである.多くのMOT手法は,Trackingアルゴリズムの独自性に重点を置いている.

Trackingで重要な課題となるのが,追跡対象が影に入った場合の対応である.一度影に入って検出できなくなった物体が,再び影から出てきたとき,影に入る以前と同じIDを付与しなければならない.近年主流になっているMOT手法の多くは,物体の視覚的特徴量を用いてこの課題に対応しようとしている.追跡対象の色や形といった見た目の特徴を保持しておく.それらの視覚的特徴量とカルマンフィルタなどによる予測を組み合わせることで,影に出入りした物体に正しいIDを付与できる確率が高まり,精度の高い追跡が実現できる.

一方で,2D-LiDAR画像は周囲の物体との距離のみを示すため,カメラ画像と比べて視覚的な特徴量が極めて少ない.したがって,本研究において視覚的特徴量を使うアルゴリズムの導入は難しいと考えられた.そこで,本研究ではByteTrack\cite{ByteTrack}と呼ばれる手法を採用した.ByteTrackは,Trackingに視覚的特徴量を用いないにも関わらず,非常に高い性能を発揮できる手法として知られている.

\section{BYTEアルゴリズム}
ByteTrack\cite{ByteTrack}は,検出にYOLOX\cite{ge2021yolox},追跡にBYTEアルゴリズムを用いたトラッカーの名称である.ここでは,ByteTrackの鍵となっているBYTEアルゴリズムについて紹介する.

一般的なMOTの追跡アルゴリズムでは,閾値よりも信頼度の低い検出を切り捨てて追跡を行う.図\ref{fig:com}は,複数人が歩くカメラ映像に対して,一般的なアルゴリズムが検出と追跡を行っている様子である.Frame \(t_1\)において,真ん中を歩く人は信頼度0.8として検出され,追跡でも赤い枠で表されている.しかし,Frame \(t_2\)では左の人の影に入って信頼度が0.4に下がり,設定された閾値を下回ってしまったため,追跡は失われてしまっている.この切り捨ては,誤検出を追跡に採用しないようにするために行われる.すべての検出を追跡に採用すると,右端の誤検出も人として追跡されてしまう.

\begin{figure}[H]
   \centering
   \begin{center}
      \centering
      \includegraphics[width=115mm]{com.png}
      \caption[人物の検出と追跡(BYTEアルゴリズムの場合)]{人物の検出と追跡(一般的なアルゴリズムの場合)\newline (\cite{ByteTrack}図2より)}
      \label{fig:com}
   \end{center} 
\end{figure}

\newpage

一方で,BYTEアルゴリズムは信頼度の低い検出も追跡に用いる.まず,すべての検出を信頼度の高い検出\(D_{high}\)と低い検出\(D_{low}\)に分ける.それらを用いて,これまでのフレームを踏まえたカルマンフィルタの予測との紐づけを行う.1回目の紐づけでは,信頼度の高い検出\(D_{high}\)のみを使って紐づけを行う.信頼度の高い検出のうち過去の追跡と結びつかなかったものは,新たな追跡対象として追加される.ここまでは従来の方法と同様である.続いて,1回目の紐づけで結びつかなかった予測と信頼度の低い検出\(D_{low}\)の紐づけを行う.ここで結びつかなかった検出は切り捨てられる.以上のような流れで紐づけを行うことで,信頼度の低い検出\(D_{low}\)も無駄にすることなく追跡が行われる.また,信頼度の低い検出\(D_{low}\)は新たな追跡対象とはならないため,誤検出が追跡につながることも避けられる.

\begin{figure}[H]
   \centering
   \begin{center}
      \centering
      \includegraphics[width=115mm]{bye.png}
      \caption[人物の検出と追跡(BYTEアルゴリズムの場合)]{人物の検出と追跡(BYTEアルゴリズムの場合)\newline (\cite{ByteTrack}図2より)}
      \label{fig:bye}
   \end{center} 
\end{figure}


\section{2D-LiDAR画像への適用}

BYTEアルゴリズムを導入したByteTrackは,MOTにおけるState-of-the-Artな手法として注目を集めている.一方で,ByteTrackはカメラ画像に対する追跡技術であり,2D-LiDAR画像に対しても同様の追跡を実現できるかどうかは不明である.したがって本研究は,2D-LiDAR画像に対するByteTrackの適用可能性を,実計測データに対する追跡を行うことで検証する.

\chapter{システムの概要}

本研究では,2D-LiDAR画像に対してByteTrackを適用することで歩行者追跡を実現する.本章ではシステムの概要について示す.

\section{2D-LiDARによる計測}

\subsection{2D-LiDARの仕様}
LiDAR (light detection and ranging)とは,光でスキャニングしながら検出物までの距離を測定する二次元走査型の光距離センサである.本研究では北陽電機の UTM-30LX\cite{lidar}(図\ref{fig:lid},表\ref{tab:lid})を使用する.周囲270度に存在する物体との距離を正確に計測することができる.

\begin{figure}[H]
   \centering
   \begin{center}
      \centering
      \includegraphics[width=60mm]{lidar.jpg}
      \caption{UTM-30LX}
      \label{fig:lid}
   \end{center} 
\end{figure}


\begin{table}[H]
   \centering
   \caption{{UTM-30LX} の仕様}
   
   \begin{tabular}{|c|l|} \hline
      光源 & 半導体レーザλ=905nm,FDAレーザ安全クラス 1\\ \hline
      電源電圧 & DC12V±10\% \\ \hline 
      電源電流 & パワーON時Max1A,定常時0.7A以下\\ \hline
      検出距離 & 検出保証値 0.1〜30m,最大検出距離 60m(出力限界値)\\ \hline
      検出体 & 最小検出物 130mm(10m):距離により変動する \\ \hline
      測距制度 & 0.1〜10m:±30mm, 10〜30m:±50mm \\ \hline
      走査角度 & 270度\\ \hline
      角度分解能 & 約0.25度(360°/1440分割)\\ \hline
      走査時間 & 25ms(モータ回転数 2400rpm)\\ \hline
      インターフェース & USB Ver2.0 FSモード(12Mbps)\\ \hline
      出力 & OUTPUT 1点 同期出力 \\ \hline
      重量 & 210g(ケーブルを除く)\\ \hline

   \end{tabular}
   \label{tab:lid}
\end{table}


\subsection{2D-LiDARデータの画像化}
2D-LiDARが取得するデータは極座標で表されるため,取得した距離データのラジアン角度を求め,直交座標系に変形して描画するこのように描画された画像を本稿では距離画像と呼ぶ.また,現在の1フレームの距離画像を,本稿では現在画像と呼ぶ.


\subsection{2D-LiDARデータの時系列画像化}
先行研究と同様に,複数フレームの距離画像を重ね合わせた画像を時系列画像と呼ぶ.時系列画像は,人間などの対象物の動きを1枚の画像で表現する,Motion History Image (MHI) \cite{bobick2001recognition}という手法を用いて作成する.MHIでは,現在のフレームに近いほど明るく,過去のフレームほど暗く描画することで,対象の動きを表す.
本研究においては,フレームが更新される度に前フレームでの時系列画像の輝度を一定量減らし,最新のフレームを重ね合わせることで時系列画像を作成する.


\newpage

第2章で述べたように,先行研究においては時系列画像を用いてクラスタリングを行っていた.時系列画像を用いることで,1人の足を複数人と検出してしまうことや,複数人の足を1人と検出してしまうことを防ぐ.本研究では,YOLOX\cite{ge2021yolox}を用いて機械学習ベースでの人物の検出を実現する.最終的には,現在画像のみでの検出を目指している.しかし,現在画像は情報量が極めて少ないため,現在画像のみでの検出は容易ではないと考えられた.よって本研究においても,まずは時系列画像を用いて検出を行った.








\section{検出手法と学習}
本節では,2D-LiDARより得られる画像から人物を検出するための,物体検出の手法と学習の流れに関して述べる.

\subsection{YOLOX}
YOLOは高速かつ高精度な物体検出手法である.YOLOX\cite{ge2021yolox}は2021年に公開された新しいモデルで,アンカーフリー構造を取り入れるなどの改良がなされ,過去のモデルよりも優れた性能を発揮している.本研究では,YOLOXを用いて人物の検出を行う.

\begin{figure}[h]
   \centering
      \begin{center}
      \centering
      \includegraphics[width=120mm]{yolox.png}
      \caption[{YOLOX}とそれ以前の手法の比較]{YOLOXとそれ以前の手法の比較(\cite{ge2021yolox}図1より)}
      \label{fig:yol}
   \end{center} 
\end{figure}



\subsection{学習データ}
今回必要になる学習データは,2D-LiDARによって記録された人物の歩行データである.本研究においては,同じ2D-LiDARを用いた全身骨格推定の研究\cite{s_b}を行っている小林研究室の須田悠介氏より,歩行の様子を記録したデータを提供していただいた.提供いただいたデータは学生7人の約2.5mの直進歩行の様子を記録したものである.2D-LiDARより得られる距離データを時系列画像や現在画像に変換し,アノテーションを行うことで学習データとして使用した.

\begin{figure}[H]
   \centering
   \begin{center}
      \centering
      \includegraphics[width=100mm]{annotation.png}
      \caption{アノテーション}
      \label{fig:ann}
   \end{center} 
\end{figure}

\subsection{学習手法}

時系列画像と現在画像のそれぞれで学習を行う.全データ10347枚のうち,9384枚を学習データとして,963枚を検証データとして使用する.

学習の評価にはAP (Average Precision)を用いる.APは,適合率を表すPrecisionと再現率を表すRecallから計算される.0-1の値をとり,1に近いほど性能が高いことを示す.物体検出の評価指標として一般的に用いられる.

また,YOLOX\cite{ge2021yolox}には,求められる速度や検出率などに応じて複数の学習済みモデルが用意されている.今回は,ByteTrackにおいて標準に用いられている,yolox\_x.pthを選択した.

\newpage
時系列画像では50エポックの学習を行った.APの最大値は0.906となった.



\begin{figure}[H]
   \centering
   \begin{center}
      \centering
      \includegraphics[width=100mm]{mhi_learn.png}
      \caption{学習曲線(時系列画像)}
      \label{fig:mhi_learn}
   \end{center} 
\end{figure}

現在画像でも同様に学習を行ったが,時系列画像と比較すると精度の向上は小さかった.APの最大値は0.783となった.

\begin{figure}[H]
   \centering
   \begin{center}
      \centering
      \includegraphics[width=100mm]{now_learn.png}
      \caption{学習曲線(現在画像)}
      \label{fig:now_learn}
   \end{center} 
\end{figure}

以上の学習によって得られたモデルを使って,検出と追跡を行う.






\chapter{実験}
本研究の手法の精度を確認し,先行研究と比較するために実験を行う.

\section{実験の概要}
\subsection{環境}

実際に人が歩く様子を記録して,本研究の手法と先行研究のそれぞれで追跡を行った.3m×3mのエリア内で,2人の人物が決められた歩行を行う.近接したすれ違いや,一方の足によって他方の足が隠れるような場面を含めた歩行とした.歩行時間は約40秒間であった.



\begin{figure}[H]
   \centering
   \begin{center}
      \centering
      \includegraphics[width=453pt]{exp.png}
      \caption{実験の様子}
      \label{fig:exp}
   \end{center} 
\end{figure}


\subsection{計測・追跡の適用}
歩行の様子は1台の2D-LiDARによって記録する.データ記録用プログラムでは,時系列画像や現在画像は動画ファイルとして出力される.この動画ファイルにByteTrackを適用することで追跡を行う.また,先行研究の手法にも同じ歩行を適用するため,2D-LiDARから取得した距離データも,テキストファイルとして出力する.この距離データを先行研究のプログラムで読み込むことで,先行研究の手法によって追跡を行う.後述する評価指標を用いて,同じ歩行における各手法の精度を比較する.

\subsection{評価指標}
Multi Object Trackingにおいては,さまざまな指標によって追跡の精度が定量的に評価される.本研究においては,その中でも特に一般的である,MOTA\cite{bernardin2008evaluating},IDF1\cite{ristani2016performance}の2つの指標を評価に用いる.

\subsubsection{MOTA}

MOTA\cite{bernardin2008evaluating}は,MOTの評価指標として広く用いられている.定義を式5.1に示す.分母は各フレームにおける正解の数の総和である.追跡対象人数とフレーム数の積によって得られる.分子は,追跡における3つのミスの和である.FP (False Positive)は偽陽性を意味し,人物検出においては人物ではない部分が誤って検出されてしまうミスを指す.FN (False Negative)は偽陰性を意味し,人物追跡においては人物であるにも関わらず検出されないミスを指す.IDswはIDSwitchの略であり,人物に付与されていたIDが変化してしまうことを指す.分母の\(g_t\)は,各フレームにおける正解データの数を表す.以上より,分数部分は正解の数に対するミスの割合を示すことになる.これを1から引くことによってMOTAが求められる.数値が1に近いほど精度が高いことを意味する.


\begin{equation}
MOTA = 1 - \frac{\sum_t(FP_t + FN_t + IDsw_t)}{\sum_tg_t}
\end{equation}

\subsubsection{IDF1}

MOTAには,追跡の持続性を評価することができないという弱点があった.IDF1\cite{ristani2016performance}は,IDFP (ID False Positive)やIDFN (ID False Negative)といったIDを基準とした数値を用いることで,MOTAの弱点を解消した評価基準である.定義を式5.2に示す.IDFPは,従来のFPに加えて,IDの付与を誤っている場合もミスとしてカウントする.同様にIDFNは,従来のFNに加えて,IDの付与を誤っている場合もミスとしてカウントする.IDTPはすべての検出や正解からIDFPもしくはIDFNを引くことで求められ,正しく検出されなおかつIDの付与も合っている検出の数を示す.これらを用いて計算されるIDF1は,正しいIDを多くのフレームで付与できているほど高い数値となる.


\begin{equation}
IDF1 = \frac{2IDTP}{2IDTP + IDFP + IDFN}
\end{equation}


\subsubsection{計算の適用}

上記の指標の計算には,Pythonのpy-motmetricsライブラリ\cite{py-mot}を利用した.2D-LiDARより得られた画像に対して,手動でアノテーションを行い正解データを作成する.各追跡手法よりアノテーションデータを取得し,正解データと照らし合わせることで評価指標が計算される.





\newpage
\section{追跡結果}

\subsection{追跡の様子}
\subsubsection{時系列画像を使った追跡}

時系列画像を使った追跡は,検出が途切れるタイミングが多いものの,最初から最後まで正しいIDを保持するることができた.一方で,2人が接近する場合に1つの検出となってしまう場面も多く見られた.すれ違う瞬間に,2人いるにもかかわらず検出が1つにまとまってしまう.一方で,再び正しい検出に戻ると,もとのIDを取り戻すことができている.

\begin{figure}[H]
   \centering
   \begin{center}
      \centering
      \includegraphics[width=60mm]{mhi_good.png}
      \caption{うまく追跡できている様子}
      \label{fig:mhi_good}
   \end{center} 
\end{figure}

\begin{figure}[H]
   \centering
   \begin{center}
      \centering
      \includegraphics[width=150mm]{mhi_miss_3.PNG}
      \caption{検出が1つになってしまう様子}
      \label{fig:mhi_miss_3}
   \end{center} 
\end{figure}



\newpage
\subsubsection{現在画像を使った追跡}
現在画像を使った追跡は,時系列画像を使った追跡と比べて安定して検出を行うことができていた.しかし,時系列画像を使った場合と同様に,2人が接近する場合には1つの検出となってしまう場合が多かった.

\vspace*{20pt}

\begin{figure}[H]
   \centering
   \begin{center}
      \centering
      \includegraphics[width=60mm]{now_good.png}
      \caption{うまく追跡できている様子}
      \label{fig:now_good}
   \end{center} 
\end{figure}

\begin{figure}[H]
   \centering
   \begin{center}
      \centering
      \includegraphics[width=453pt]{now_miss_3.png}
      \caption{検出が1つになってしまう様子}
      \label{fig:now_miss_3}
   \end{center} 
\end{figure}

\newpage
\subsubsection{先行研究をの手法を使った追跡}
先行研究の手法による追跡は,一方が他方の影に入る場面でIDの保持に失敗してしまった.陰に入ったときに検出が失われ,検出が復活した際には違うIDが付与されてしまっている.
その一方で,2人が近接する場合には非常に高い検出精度を発揮した.

\vspace*{20pt}


\begin{figure}[H]
   \centering
   \begin{center}
      \centering
      \includegraphics[width=453pt]{kal_miss_3.png}
      \caption{遮蔽によってIDが切り替わってしまう様子}
      \label{fig:kal_miss_3}
   \end{center} 
\end{figure}

\begin{figure}[H]
   \centering
   \begin{center}
      \centering
      \includegraphics[width=60mm]{kal_cross.png}
      \caption{接近しても正しく検出できている様子}
      \label{fig:kal_cross}
   \end{center} 
\end{figure}


\newpage
\subsection{精度評価}

続いて,上記の評価基準に基づいて定量評価を行った.結果を表\ref{tab:score}に示す.


\begin{table}[H]
   \centering
   \caption{精度評価}
      
      \begin{tabular}{|l||r|r|r||r|r|} \hline
            & FP & FN & IDsw & MOTA & IDF1\\ \hhline{|=|=|=|=|=|=|}
         本研究(時系列画像) & 15 & 828 & 2 & 0.637 & 0.780\\ \hline 
         本研究(現在画像) & 10 & 114 & 6 & 0.944 & 0.962\\ \hline
         先行研究 & 105 & 3 & 2 & 0.953 & 0.689\\ \hline
         
      
      \end{tabular}
   \label{tab:score}
\end{table}


MOTAは,先行研究の手法が最も高くなった.遮蔽が発生する場面でIDスイッチが発生したものの,その他の場面では非常に安定した追跡を実現していた.一方で,現在画像を使った手法においてもMOTAは非常に高い数値となった.2人が近接する場合を除いては,極めて安定した追跡が行えた.時系列画像を使った手法のMOTAが低くなっているのは,FNの回数が極端に多いことに由来すると考えられる.時系列画像を使った追跡では,検出が途切れる場合が多いため,FNの回数が多くなってしまった.これは,検出の精度を向上させることで解決できると考えられる.

IDF1は,現在画像を使った手法が最も高くなった.現在画像を使った手法は,最初から最後まで同じIDを維持できており,これがIDF1の良好な数値に繋がったと考えられる.途中でIDが切り替わってしまった先行研究の手法のIDF1は,最も低い数値となっている.追跡の持続性という観点では,本研究の手法が有用であることがわかった.


また,本研究の手法を使った追跡は,最初から最後まで同じIDを保持できていたにも関わらず,IDスイッチが複数回記録されている.これは,2人が接近する場合に検出の精度が落ち,異なるIDによる追跡が数フレーム発生したことによると考えられる.



\newpage
\section{追加実験}
前項での結果を踏まえて,追跡の支障となりやす場面に関してさらに実験を行う.何らかの遮蔽物によって,一時的に対象が影に入る場合の対応は,複数物体追跡において大きなテーマとなる.先行研究においても,遮蔽が発生する場面での追跡が大きな課題となっていた.実際に今回の実験でも,先行研究の手法は遮蔽が発生する場合にIDスイッチが発生してしまっていた.その一方で,本研究の手法ではIDスイッチが発生することはなかった.

遮蔽が発生する場合に関して,更に実験を行った.意図的に遮蔽が発生する状況を作り,5往復の歩行を行う.これを4セット行い,計40回の遮蔽がある歩行を記録した.遮蔽物の横幅はおよそ20cmである.各手法を適用して追跡を行い,IDスイッチの回数を記録した.




\begin{figure}[H]
   \centering
   \begin{center}
      \centering
      \includegraphics[width=120mm]{occ_exp.png}
      \caption{追加実験の様子}
      \label{fig:exp}
   \end{center} 
\end{figure}



結果を表\ref{tab:occ_score}に示す.


\begin{table}[H]
      \centering
      \caption{遮蔽がある状況におけるID保持}
      
      \begin{tabular}{|l||r|r||r|} \hline
            &成功&失敗& 成功率\\ \hhline{|=|=|=|=|}
         本研究(時系列画像) &32&8&0.800 \\ \hline 
         本研究(現在画像) & 39&1&0.975\\ \hline
         先行研究 & 2&38&0.050 \\ \hline
         
      
      \end{tabular}
      \label{tab:occ_score}
\end{table}

先行研究の手法においては,ほとんどの試行においてIDが切り替わってしまった.追跡にカルマンフィルタを用いているため,先行研究ではこのような遮蔽によるIDスイッチが多発し,安定した追跡の障壁となっていた.


\begin{figure}[H]
   \centering
   \begin{center}
      \centering
      \includegraphics[width=453pt]{occ_kal.png}
      \caption{IDが切り替わってしまう様子}
      \label{fig:occ_kal}
   \end{center} 
\end{figure}

対して,本研究の手法はIDを正しく保持できる場合が多かった.追跡に現在画像を用いた場合には,特にIDスイッチが少なくなった.時系列画像においては,遮蔽に入る前の検出精度が低い場合に,IDスイッチが発生していることが多かった.検出精度が低いと追跡が細切れになり,進行方向と速度の予測が不十分になるために,遮蔽後のID付与に失敗していると考えられる.これは検出精度の改善で解消されると考えられる.

\begin{figure}[H]
   \centering
   \begin{center}
      \centering
      \includegraphics[width=453pt]{occ_mhi.png}
      \caption{IDを保持できている様子}
      \label{fig:occ_mhi}
   \end{center} 
\end{figure}

今回の実験から,本研究の手法は遮蔽が発生する場面でも精度の高い追跡が行えることが確認できた.










\chapter{考察}

\section{提案手法の性能}
本研究では,深層学習を用いて2D-LiDARでの歩行者追跡の高度化を目指し,システムの実装を行った.歩行者同士が近接する場合を除けば,概ね正しい追跡が行えることを確認した.遮蔽やすれ違いなどを含めた40秒間の歩行で,IDを維持し続けることができた点は評価できる.

一方で,複数人が近接する場面での検出精度には課題が見られた.データセットの量と質の改善や学習の設定の見直しなどによって,検出の精度を上げていくことが求められる.特に,学習データの質に関しては大きな改善の余地がある.現在の学習データは,2D-LiDARへ向かって直進する歩行のみであり,バリエーションが極めて少ない.また,2D-LiDARから2m-4.5mの距離の間での歩行を記録して学習データとしているため,2D-LiDARとの距離が5mを超えると性能が大幅に落ちる.学習データのバリエーションを増やすことで,検出精度の向上させる必要がある.

\newpage
\section{先行研究との比較}
\subsection{追跡精度}
精度の面で,先行研究よりも高い性能を発揮できると言えるような結果を得ることはできなかった.MOTAやIDF1などの評価指標において,先行研究と同程度のスコアを発揮したものの,先行研究を上回ることはできていない.特に,歩行者同士が近接する場合の追跡精度は,先行研究の方が高い精度を発揮した.

一方で,遮蔽が発生する場面では,本研究の手法が極めて有用であることが確認できた.複数人が歩行する場面では,他の人物の足によって頻繁に遮蔽が発生するため,先行研究では追跡精度が低下していた.この課題は解消に大きく近づいたと言える.


\subsection{処理速度}
先行研究の問題点のひとつとして,追跡対象が増える場合に処理速度が落ちる点があげられていた.現状の本研究の手法は,リアルタイムでの追跡を行っていない.動画ファイルにByteTrackを適用する際は,およそ20fpsで処理が行われている.リアルタイムでの追跡を行う場合も同程度の処理速度を見込んでいる.リアルタイムでの追跡の実装と処理速度の面での比較は,今後の課題としたい.

\section{時系列画像と現在画像の比較}
今回の実験では,時系列画像を用いた場合と現在画像を用いた場合のそれぞれで追跡を行った.時系列画像の方が情報量が多く,高い精度での追跡が行えると予想していた.しかし,実際には現在画像を用いた場合の方が良い数値を得ることが出来た.時系列画像の方が歩行の状態を細かく表現できるために,学習データのバリエーションの少なさの影響を強く受け,検出がうまくいかなかったことが原因と考えられる.学習データの改善によって検出精度の向上を目指したい.一方で,現在画像でも先行研究と同程度の性能を得られたことは,2D-LiDARが動く環境での追跡へ向けて大きな進歩と言える.


\chapter{おわりに}
 \section{まとめ}
 足元の高さに設置した2D-LiDARを使った歩行者追跡の,深層学習を用いた高度化について提案した.カメラ画像に対する複数物体追跡の中で優れた性能を発揮しているByteTrackのアルゴリズムを,2D-LiDAR画像に対する複数物体追跡に応用することで,従来の手法における問題点の解消を目指した.本稿では,実計測データに対して提案手法を用いて追跡を行うことで,従来の手法と比較すると同時に,本研究の提案手法の有用性を確認した.

複数人の歩行に対して,多くの場面で安定した追跡が行えることを確認した.特に,最初から最後まで同じIDを付与し続けることができた点は,追跡の持続性という観点から高く評価できる.従来の手法で課題となっていた追跡対象が遮蔽される状況においても,正しく追跡を保持できることを確認した.一方で,複数人が極めて接近する歩行など,追跡がうまくいかない場合があることも確認された.追跡に失敗する事象の多くは,検出精度の低さが原因であると考察した.

 \section{今後の課題}
 複数人が近接する場合に追跡精度が落ちるのは,検出の精度が不足しているためであると考えれる.学習に用いるデータセットの改善や,学習の設定の改善などによって,検出精度の改善を目指していきたい.また,安定した追跡性能を実現したうえで,移動ロボットに搭載して計測する場合を想定し,リアルタイムでの追跡や2D-LiDARが動く場合の追跡などにも挑戦していきたい.


 



\newpage

% 参考文献:bibtexを使う場合


\bibliographystyle{junsrt}
\bibliography{books}
%
%\bibliographystyle{aiseplain} % bstファイル名
%\bibliography{refs} % bibファイル名


% 付録
\appendix % 以下,付録

% 索引. mendexかmkindexで作成
\printindex

%目次にIndexを表示させる場合はmendexかmkindex実行後に作成される
% hogehoge.idmファイルのtheindex環境の中で下記命令を置く
%\addcontentsline{toc}{chapter}{Index}

\end{document}

