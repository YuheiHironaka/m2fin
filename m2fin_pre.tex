\documentclass{jarticle}
\usepackage{mabst}
\usepackage[dvipdfmx]{graphicx}
\usepackage{latexsym}
\usepackage{here}

\usepackage{comment}
\usepackage{cite}
\usepackage{url}
\usepackage{hhline}
\usepackage{makecell}
\usepackage{titlesec}
\usepackage{indentfirst}

% 見出しの前後のスペースをさらに詰める
\titlespacing*{\section}{0pt}{0.4ex}{0.3ex}
\titlespacing*{\subsection}{0pt}{0.3ex}{0.2ex}
\titlespacing*{\subsubsection}{0pt}{0.3ex}{0.2ex}

% 段落間の余白をゼロにする
\setlength{\parskip}{0pt}

% 画像と文章の間隔をさらに詰める
\setlength{\intextsep}{4pt plus 1pt minus 1pt}      % [H] の図の上下
\setlength{\textfloatsep}{4pt plus 1pt minus 1pt}  % ページの上下にある図と本文の間
\setlength{\abovecaptionskip}{1pt}   % 図とキャプションの間
\setlength{\belowcaptionskip}{1pt}   % キャプションと下の本文の間



\begin{comment}

\end{comment}


\title{2D-LiDARを用いた足元計測に基づく人物行動の理解}



\newcommand{\ID}{24MM322}
\newcommand{\Name}{廣中 優平}
\newcommand{\Teacher}{小林 貴訓}



\begin{comment}




\end{comment}



\begin{document}
\maketitle

\section{はじめに}
% 人の動きを計測することは,様々な場面で非常に重要である.近年では公共空間で人と共存するロボットも多く登場し,人の位置を正しく検出・追跡したり,個人の振る舞いを理解することの重要性は一層増している.現在行われている人物計測の多くはカメラ画像を用いており,詳細な計測を行うことができるものの,公共空間における計測ではプライバシーの問題が懸念される.そこで本研究では,レーザセンサである2D-LiDARを用いることで,少ない情報量から人物の行動を計測するシステムの実装を目指す.

% 人物行動の計測では,人物の位置を継続的に把握することや,人物の振る舞いを理解することが求められる.以前より我々の研究室では,2D-LiDARを用いた人物の検出・追跡や骨格推定に取り組んできた.足元に設置した2D-LiDARによる計測データの複数フレームを重ねて人物の足元の動きを表現した時系列画像に対して,クラスタリング手法による人物の検出とカルマンフィルタによる追跡を行うことで,精度の高い歩行者追跡が可能となっている\cite{hasan2021person}.また,同様の足元計測画像に,Kinectから取得した骨格情報を組み合わせて深層学習を行うことで,足元計測のみで歩行者の3次元全身骨格をおおよそ推定できる枠組みを実現している\cite{suda2023}.これらの技術はどれも非常に有用なものであるが,複数人が歩行して遮蔽が多発する場合に追跡性能が落ちる点,クラスタリングの処理速度が遅くリアルタイムでの追跡に向かない点,それぞれが個別に実装されている点など,実用に向けていくつかの問題点があった.本研究では,歩行者追跡におけるいくつかの問題を解消すると共に,既存の骨格推定手法との統合を図ることで,2D-LiDARを用いた実用的な人物行動計測システムを目指す.


人物行動計測は,様々な場面で非常に重要である.近年では公共空間で人と共存するロボットも多く登場し,複数人の正確な追跡や個々の振る舞いの理解の重要性は一層増している.現在の人物計測の多くはカメラ画像を用いており,詳細な計測が行えるものの,公共空間での計測ではプライバシーの問題が懸念される.そこで本研究では,距離のみを測るセンサである 2D-LiDARを用いることで,少ない情報量から人物の行動を計測するシステムの実装を目指す.

人物行動計測では,人物の位置の継続的な把握や,人物の振る舞いの理解が求められる.以前より我々の研究室では,2D-LiDARを用いた人物の検出・追跡や骨格推定に取り組んできた.足元に設置した2D-LiDARによる計測データの複数フレームを重ねて人物の足元の動きを表現した時系列画像に対して,クラスタリングによる人物検出とカルマンフィルタによる追跡の組み合わせで,高精度な歩行者追跡が可能としている\cite{hasan2021person}.また,同様の足元計測画像に,Kinectから取得した骨格情報を組み合わせて深層学習を行うことで,足元計測のみで歩行者の3次元全身骨格をおおよそ推定できる枠組みを実現している\cite{suda2023}.これらの技術はどれも非常に有用であるが,複数人が歩行する公共空間においては遮蔽が多発して追跡性能が落ちる点,クラスタリングの処理速度が遅くリアルタイムでの追跡に向かない点,それぞれが個別に実装されている点など,実用に向けていくつかの問題点があった.本研究では,歩行者追跡における問題を解消し,既存の骨格推定手法との統合によって,2D-LiDARによる足元計測データのみから歩行者の歩行動作をリアルタイムに可視化する計測システムの開発を行う.



% \begin{figure}[H]
%      \centering
%      \includegraphics[width=230pt]{image_Sensing_neo.png}
%      \caption{骨格推定の様子}
%      \label{fig:sensing}
% \end{figure}



%本研究では,カメラ画像に対する複数物体追跡の技術であるMOT (Multi Object Tracking)を2D-LiDAR画像に応用することで,前述の問題点の解消を目指す.これまでに提案されてきたMOTアルゴリズムの中で,視覚的特徴量を使わないことからByteTrack\cite{zhang2022bytetrack}のアルゴリズムを導入することとした.これは,2D-LiDAR画像はカメラ画像と比べて極めて情報量が少なく,特徴を抽出することが難しいと考えられるためである.


\section{提案手法}
本システムは,計測と結果の描画を行う計測用マシンと,深層学習モデルの推論などの重い計算処理を行う計算用マシンの2台の連携によって動作する(図\ref{fig:system_flow}).

\begin{figure}[H]
     \centering
     \includegraphics[width=230pt]{system_flow_3.png}
     \caption{提案システムの構成}
     \label{fig:system_flow}
\end{figure}

% \begin{figure}[H]
%      \centering
%      \includegraphics[width=220pt]{pre_mth.png}
%      \caption{システムの概要}
%      \label{fig:sensing}
% \end{figure}
\subsection{距離データの画像処理}

2D-LiDARによる計測データは,計算用マシンへ距離データのまま送信される.計算用マシンでは,受信した距離データを直交座標系へ変換し,2D-LiDARを中心とした2次元画像を作成する.さらに,対象の動きを表現するために,Motion History Image (MHI)を作成する.MHIは,複数のフレームを重ね合わせ,現在のフレームに近いほど明るく,過去のフレームほど暗く描画する手法である.これにより,1枚の画像で計測対象の動きを表現することができ,検出や骨格推定の精度向上が期待できる.
\subsection{検出・追跡}
歩行者の追跡には,カメラ画像処理で高い性能を示すByteTrackを応用した\cite{ByteTrack}.2D-LiDAR画像はカメラ画像に比べて視覚的特徴量が少ないため,追跡対象の外見の特徴に依存しないByteTrackのアルゴリズムが適していると考えた.
検出にはYOLOXを用いる.従来のYOLOXはカメラ画像中の人物や物体を検出できるような学習がなされているため,本システムでは新たな転移学習が必要となった.実際に歩行者の足元の様子を2D-LiDARで計測し,時系列画像を作成してアノテーションを行うことで学習データとした.2499枚の学習データ,650枚のテストデータを用いて学習を行い,十分な検出精度が得られることを確認した.
追跡にはBYTEアルゴリズムを用いる.これは,信頼度の高い検出結果だけでなく,低い検出結果も破棄せずに追跡に応用する手法である.遮蔽が発生した場合でも,低い信頼度の検出との紐付けを行うことで,追跡を継続できる可能性が高まる.これにより,遮蔽が多い環境でもIDを維持しやすい,持続性の高い追跡を実現できる.
\subsection{骨格推定}
検出・追跡によって得られたバウンディングボックスの位置に基づき,計測域全体のMHIから人物領域を切り出す.この際,単に追跡結果のボックス内を切り出すのではなく,広めの領域で切り出してから重心計算を行うことで,足全体が確実に収まるように位置を補正して100×100ピクセルの画像を生成する.
骨格推定モデルには,画像認識分野で広く用いられるResNet-18をベースに,2D-LiDAR計測データから骨格を推定できるように学習されたモデルを使用する\cite{suda2023}.学習データは204466枚,テストデータは16572枚であり,この学習データには歩行者の正面・背後・左右から記録した4方向の直進歩行のみが含まれている.
入力されたMHIから頭や手足など全身25箇所の位置を3次元座標として出力する.

\subsection{システムの統合と可視化}
前節までに述べた処理は,すべて計算用マシンで行われる.計算用マシンではマルチスレッド処理を導入した.負荷の異なる処理を並列化することで,重い処理による遅延が全体に及ぶのを防ぎ,リアルタイム動作を可能にしている.

各処理の結果は計測用マシンへ送り返される.計測用マシンにて保持している2D-LiDAR点群と受信した結果のタイムスタンプを照合することで,処理時間によるズレを補正した結果出力を行う.2D-LiDARの点群データ上に,推定された骨格モデルを重ねて3次元出力することで,空間内の状況を直感的に把握できる表示を実現した.
また,形状が大きく崩れた骨格が推定された場合には,その結果を破棄し,前フレームの骨格を維持する補完処理を導入した.歩行の様子がわかりやすい滑らかな可視化を実現している.




% \section{提案手法}
% 2D-LiDARで計測した複数歩行者の検出・追跡と骨格推定をリアルタイムで行う,統合的な計測システムを提案する.
% \subsection{システム構成}
% 本システムは,計測と結果の描画を行う計測用マシンと,深層学習モデルの推論などの重い計算処理を行う計算用マシンの2台の連携によって動作する(図\ref{fig:system_flow}).
% 計測用マシンが,2D-LiDARから取得した距離データを計算用マシンへ送信する.計算用マシンでは,画像化,検出・追跡,骨格推定といった計算負荷の異なる処理を行う.これらの処理時間が異なるため,それぞれを並列に動作させるマルチスレッド処理を採用した.これにより,重い処理による遅延がシステム全体に及ぶのを防ぎ,リアルタイム動作を実現している.
% \begin{figure}[H]
%      \centering
%      \includegraphics[width=190pt]{system_flow.png}
%      \caption{提案システムの構成}
%      \label{fig:system_flow}
% \end{figure}
% % \begin{figure}[H]
% %      \centering
% %      \includegraphics[width=220pt]{pre_mth.png}
% %      \caption{システムの概要}
% %      \label{fig:sensing}
% % \end{figure}
% \subsection{距離データの画像処理}
% 計算用マシンでは,受信した距離データを直交座標系へ変換し,2D-LiDARを中心とした2次元画像を作成する.さらに,対象の動きを表現するために,Motion History Image (MHI)を作成する.MHIは,複数のフレームを重ね合わせ,現在のフレームに近いほど明るく,過去のフレームほど暗く描画する手法である.これにより,1枚の画像で計測対象の動きを表現することができ,検出や骨格推定の精度向上が期待できる.
% \subsection{検出・追跡}
% 歩行者の追跡には,カメラ画像処理で高い性能を示すByteTrackを応用した\cite{ByteTrack}.2D-LiDAR画像はカメラ画像に比べて視覚的特徴量が少ないため,追跡対象の外見の特徴に依存しないByteTrackのアルゴリズムが適していると考えた.
% 検出にはYOLOXを用いる.従来のYOLOXはカメラ画像中の人物や物体を検出できるような学習がなされているため,本システムでは新たな転移学習が必要となった.実際に歩行者の足元の様子を2D-LiDARで計測し,時系列画像を作成してアノテーションを行うことで学習データとした.2499枚の学習データ,650枚のテストデータを用いて学習を行い,十分な検出精度が得られることを確認した.
% 追跡にはBYTEアルゴリズムを用いる.これは,信頼度の高い検出結果だけでなく,低い検出結果も破棄せずに追跡に応用する手法である.遮蔽が発生した場合でも,低い信頼度の検出との紐付けを行うことで,追跡を継続できる可能性が高まる.これにより,遮蔽が多い環境でもIDを維持しやすい,持続性の高い追跡を実現できる.
% \subsection{骨格推定}
% 検出・追跡によって得られたバウンディングボックスの位置に基づき,計測域全体のMHIから人物領域を切り出す.この際,単に追跡結果のボックス内を切り出すのではなく,まず広めの領域で切り出してから重心計算を行うことで,足全体が確実に収まるように位置を補正して100×100ピクセルの画像を生成する.
% 骨格推定モデルには,画像認識分野で広く用いられるResNet-18をベースに,2D-LiDAR計測データから骨格を推定できるように学習したモデルを使用する\cite{suda2023}.入力されたMHIから,頭や手足など全身25箇所の位置を3次元座標として出力する.
% \subsection{統合的な描画}
% 得られた追跡結果と骨格座標は計測用マシンへ送り返される.この際,受信した結果と計測用マシンにて保持してある2D-LiDAR点群のタイムスタンプを照会することで,処理時間によるズレを補正した結果出力を行えるようにしている.2D-LiDARの点群データ上に,推定された骨格モデルを重ねて3次元表示することで,空間内の状況を直感的に把握できる.
% また,形状が大きく崩れた骨格が推定された場合には,その結果を破棄し,前フレームの骨格を維持する補完処理を導入した.これにより滑らかな可視化が実現される.
\section{実験}
\subsection{検出・追跡の性能評価}
歩行者検出・追跡の性能を確認する.
3m×3.5mのエリア内において,与えられたパターンに従って複数人が歩行し,その様子を足元の高さに設置した1台の2D-LiDARによって計測する(図\ref{fig:walking}).
どちらの手法も,多くのフレームで正しく人物を検出できた.一方で,追跡性能には差が見られた.従来の手法では,他の人の影に入った時や複数人が接近した時に検出が失われた後,違うIDが付与されてしまう場合が多かった.ByteTrackを用いた提案手法では,検出に失敗するフレームがあったとしても,正しいIDを再び付与することができた.
 % 複数物体追跡において一般的な評価指標であるIDF1を用いて定量的な評価を行った\cite{bernardin2008evaluating}.結果を下に示す.
% 複数物体追跡において一般的な評価指標であるMOTA・IDF1・HOTAを用いて定量的な評価を行った\cite{bernardin2008evaluating}.
\begin{figure}[H]
     \centering
     \includegraphics[width=170pt]{byte3.png}
     \caption{複数人の歩行における検出・追跡}
     \label{fig:walking}
\end{figure}
一般的な評価指標を用いて定量評価を行う\cite{bernardin2008evaluating}.
追跡の持続性を重点的に評価するIDF1において,従来の手法と比べて非常に高い数値を得ることができた(表\ref{tab:score}).検出が失われる場面でもIDの切り替えが少なかったことに起因していると考えられる.このような持続性の高い追跡は,長期的な人流の解析や特定の人物の振る舞いの理解に向けて非常に重要である.

\begin{table}[H]
    \centering
    \caption{検出・追跡の定量評価}
    \label{tab:score}
    \begin{tabular}{|l||c|c|}
        \hline
        & MOTA & IDF1  \\ \hhline{|=|=|=|}
        \makecell{クラスタリング \\ +カルマンフィルタ } & 0.8401  & 0.5342 \\
        \hline
        \makecell{ByteTrack } & 0.8374 & \textbf{0.9117}  \\
        \hline
    \end{tabular}
\end{table}



\subsection{骨格推定の性能評価}
本節では,3次元全身骨格推定を含めた統合システム全体について評価を行う.

単純な直進歩行に対して統合システムを適用した.
実際の歩行と同じように手足を振る骨格が推定され,検出・追跡と同時にリアルタイムで動作することを確認した.
学習データに近い直進歩行に対しては,2D-LiDARの足元情報のみから全身の動きをおおよそ推定できることがわかった.


複数人が自由に歩行するシーンに対して,検出・追跡と骨格推定を統合した3次元空間表示を試みた(図\ref{fig:3d_multi}).推定された骨格を2D-LiDARで取得した点群に重ねることで,カメラで見ているような直感的でわかりやすい計測が行えることを確認した.一方で,曲線的な歩行や斜め方向の歩行に対しては,骨格が崩れる場面も見られた.これは,学習データが4方向の直進歩行に限定されているためであると考えられる.

% \begin{figure}[H]
%      \centering
%      \includegraphics[width=180pt]{ske_one_pre.png}
%      \caption{直進歩行の骨格推定}
%      \label{fig:ske}
% \end{figure}



\begin{figure}[H]
     \centering
     \includegraphics[width=170pt]{pre_3d_2.png}
     \caption{複数人の歩行における骨格推定と3次元描画}
     \label{fig:3d_multi}
\end{figure}
\section{おわりに}
本稿では,2D-LiDARによる足元計測データを利用して,歩行者追跡と3次元全身骨格推定を行うシステムを提案した.ByteTrackの導入により,従来の手法よりも持続性の高い歩行者追跡を,リアルタイムで行うことが可能になった.また,歩行者追跡と同時に全身骨格の推定が行えることも確認した.

今後は,多様な歩行動作に対応するため,学習データの拡充などによる骨格推定精度の向上が求められる.また,実環境での使用に向けて,より広範囲かつ多人数が混在する環境での性能について評価する必要がある.

\footnotesize
\renewcommand{\pubrefname}{外部発表}
\begin{thepubbibliography}{1}
\pubbibitem{latex1}廣中優平,鈴木亮太,小林貴訓.2D-LiDARによる足元計測に基づくByteTrackを用いた歩行者追跡.第30回画像センシングシンポジウム,2024.
\end{thepubbibliography}





\bibliographystyle{junsrt}
\bibliography{books_pre}

\end{document}

