\documentclass{jarticle}
\usepackage{mabst}
\usepackage[dvipdfmx]{graphicx}
\usepackage{latexsym}
\usepackage{here}

\usepackage{comment}
\usepackage{cite}
\usepackage{url}
\usepackage{hhline}
\usepackage{makecell}



\begin{comment}

\end{comment}


\title{2D-LiDARを用いた足元計測に基づく人物行動の理解}



\newcommand{\ID}{24MM322}
\newcommand{\Name}{廣中 優平}
\newcommand{\Teacher}{小林 貴訓}



\begin{comment}




\end{comment}



\begin{document}
\maketitle

\section{はじめに}
人の動きを計測することは,様々な場面で非常に重要である.近年では公共空間で人と共存するロボットも多く登場し,人の位置を正しく検出・追跡したり,個人の振る舞いを理解することの重要性は一層増している.
現在行われている人物計測の多くは,カメラ画像を用いて行われている.カメラ画像は情報量が多く,正確な計測を行うことができる.
一方で,公共空間での撮影においては,不特定多数の人物の顔などが記録されるため,プライバシーの問題が発生する.


以前より我々の研究室では,足元の高さに設置した2D-LiDARを用いることで,プライバシーに配慮した人物計測を目指してきた\cite{hasan2021person}.2D-LiDARによる足元計測データの複数フレームを重ねて人物の動きを表現した時系列画像に対して、クラスタリング手法による人物の検出とカルマンフィルタによる追跡を行うことで、精度の高い歩行者追跡を実現している。また、2D-LiDAR計測データから歩行者の全身骨格を推定する研究も行われてきた\cite{suda_d}。先述の2D-LiDARによる時系列画像に、Kinectから取得した骨格情報を組み合わせて深層学習を行うことで、足元計測のみで歩行者の3次元全身骨格をおおよそ推定できる枠組みを実現している。これらの技術はどれも非常に有用なものであるが、複数人が歩行して遮蔽が多発する場合に追跡性能が落ちる点、クラスタリングの処理速度が遅くリアルタイムでの追跡に向かない点、それぞれが個別に実装されている点など、実用に向けていくつかの問題点があった。




% \begin{figure}[H]
%      \centering
%      \includegraphics[width=230pt]{image_Sensing_neo.png}
%      \caption{骨格推定の様子}
%      \label{fig:sensing}
% \end{figure}



%本研究では,カメラ画像に対する複数物体追跡の技術であるMOT (Multi Object Tracking)を2D-LiDAR画像に応用することで,前述の問題点の解消を目指す.これまでに提案されてきたMOTアルゴリズムの中で,視覚的特徴量を使わないことからByteTrack\cite{zhang2022bytetrack}のアルゴリズムを導入することとした.これは,2D-LiDAR画像はカメラ画像と比べて極めて情報量が少なく,特徴を抽出することが難しいと考えられるためである.

本研究では,歩行者追跡におけるいくつかの問題を解消すると共に,既存の骨格推定手法との融合を図っていく.2D-LiDARによって足元の様子を計測し,歩行者の検出と追跡を行う.更に,検出された人物の歩行の様子から三次元骨格を推定する.この一連の流れを同時に実行できる,実用性の高いシステムを目指す.



\section{提案手法}

% 歩行者追跡における諸問題の解消に向けて,カメラ画像に対する複数物体追跡の技術であるMOT (Multi Object Tracking)を導入する.
% これまでに提案されてきたMOTアルゴリズムの中で,視覚的特徴量を使わないByteTrack\cite{zhang2022bytetrack}を採用した.従来のByteTrackと同様に,検出にはYOLOX\cite{ge2021yolox},追跡にはBYTEアルゴリズムを用いる.BYTEアルゴリズムは追跡対象が遮蔽される場面の多い環境に適しており,複数人が歩行する場合の追跡性能向上が期待される.また,従来のクラスタリングを用いた手法と比べて処理速度が高く,リアルタイムでの追跡が可能になると考えられる.


%これは,2D-LiDAR画像はカメラ画像と比べて極めて情報量が少なく,特徴を抽出することが難しいためである.
足元に設置した2D-LiDARを用いて,歩行者追跡と全身骨格推定を同時に行うシステムを提案する.




2D-LiDARによって取得した距離データを直交座標系へ変換した上で,対象の動きを1枚の画像で表現するMHI(Motion History Image)を作成する.MHIは,複数のフレームを重ね合わせ,現在のフレームに近いほど明るく,過去のフレームほど暗く描画することで,対象の動きを1枚の画像で表現する手法である.

歩行者の検出・追跡においては,カメラ画像に対する複数物体追跡の技術であるMOT (Multi Object Tracking)を導入した.
MOTアルゴリズムの中で,視覚的特徴量を使わないByteTrack\cite{ByteTrack}を採用する.ByteTrackで用いられるBYTEアルゴリズムは,追跡対象が遮蔽されることの多い環境に適しており,複数人が歩行する場合の追跡性能向上が期待される.また,従来のクラスタリングとカルマンフィルタを用いた手法と比べて処理速度が高く,リアルタイムでの追跡が可能になると考えられる.



% 1フレームのみの一般的な画像と比べて動きに関する情報量が増え,高い精度での人物検出ができる.



また,検出・追跡のために作成したMHIを流用して,歩行者の三次元骨格推定を同時に行う.YOLOXによる検出に基づいて,計測域全体の画像から人物と思われる部分のMHIを切り出し,学習されたモデルを使って骨格を推定する.推定された骨格は,2D-LiDAR点群とともに3次元空間に描画することで,カメラで撮影したような直感的な見た目の人物計測を目指す.

%今回は新たに学習は行わず,これまでの研究で得られたモデルを使用する\cite{suda_d}.


一連の処理は,計測と描画を行う手元のPCと,重い計算処理を行う計算用マシンの2台を用いて行う.MHI作成,検出・追跡,骨格推定といった計算負荷の異なる処理を,計算用マシン上でマルチスレッド化して並列に実行することで,リアルタイムでの動作を実現した.


\begin{figure}[H]
     \centering
     \includegraphics[width=220pt]{pre_mth.png}
     \caption{システムの概要}
     \label{fig:sensing}
\end{figure}




%検出器には従来のByteTrackと同様にYOLOX\cite{ge2021yolox}を用いる.2D-LiDAR画像中の人物を検出するために,実際に歩行の様子を2D-LiDARで撮影した画像からMHIを作成し学習を行う.

%実際に歩行を追跡する際は,あらかじめ記録したデータに対して追跡を行う他に,リアルタイムで追跡を行うことも可能とした.従来の手法と合わせて,次章にて性能比較を行う.


%これまでに提案されてきたMOTアルゴリズムの中で,視覚的特徴量を使わないことからByteTrack\cite{zhang2022bytetrack}のアルゴリズムを導入することとした.これは,2D-LiDAR画像はカメラ画像と比べて極めて情報量が少なく,特徴を抽出することが難しいと考えられるためである.



\section{実験}

 \subsection{検出・追跡の性能評価}
まず,歩行者検出・追跡の性能を確認する.
実際に複数人が歩行する環境を計測し,検出と追跡を行った.3m×3.5mのエリア内において,与えられたパターンに従って複数人が歩行し,その様子を足元の高さに設置した1台の2D-LiDARによって測定する(図\ref{fig:walking}).



どちらの手法も,多くのフレームでは正しく人物を検出できた.一方で,追跡性能には差が見られた.従来の手法では,他の人の影に入った時や複数人が接近した時に検出が失われた後,違うIDが付与されてしまう場合が多かった.ByteTrackを用いた提案手法では,検出に失敗するフレームがあったとしても,正しいIDを再び付与することができた.


 % 複数物体追跡において一般的な評価指標であるIDF1を用いて定量的な評価を行った\cite{bernardin2008evaluating}.結果を下に示す.

% 複数物体追跡において一般的な評価指標であるMOTA・IDF1・HOTAを用いて定量的な評価を行った\cite{bernardin2008evaluating}.
\begin{figure}[H]
     \centering
     \includegraphics[width=220pt]{byte3.png}
     \caption{複数人の歩行における検出・追跡}
     \label{fig:walking}
\end{figure}







複数物体追跡における一般的な評価指標を用いて追跡性能の評価を行う\cite{bernardin2008evaluating}.検出の正確性を重視するMOTAでは,従来の手法の方が僅かに高い数値となった.提案手法のYOLOXを用いた人物検出では,遮蔽などで足の形がはっきり見えていない場面などで検出できない場合があるため,MOTAが伸びなかったと考えられる.

一方で,追跡の持続性を重点的に評価するIDF1では,従来の手法と比べて非常に高い数値を得ることができた.検出が失われる場面でもIDの切り替えが少なかったことに起因していると考えられる.このような持続性の高い追跡は,長期的な人流の解析や特定の人物の振る舞いの理解に向けて非常に重要である.


\begin{table}[H]
    \centering
    \caption{検出・追跡の定量評価}
    \label{tab:score}
    \begin{tabular}{|l||c|c|}
        \hline
        & MOTA & IDF1  \\ \hhline{|=|=|=|}
        \makecell{クラスタリング \\ +カルマンフィルタ } & 0.8401  & 0.5342 \\
        \hline
        \makecell{ByteTrack } & 0.8374 & \textbf{0.9117}  \\
        \hline
    \end{tabular}
\end{table}



\subsection{骨格推定の性能評価}
次に,三次元全身骨格推定を統合した計測システムに関して評価を行う.

まず,学習データに含まれる単純な直進歩行に対して統合システムを適用した.
実際の歩行に近い状態の手足の振りが再現された骨格が推定され,検出・追跡と同時にリアルタイムで動作することを確認した(図\ref{fig:ske}).
モデルの学習データに含まれているような直進歩行に対しては,2D-LiDARの足元情報のみから全身の動きをおおよそ推定できることがわかった.


次に,複数人が自由に歩行するシーンに対して,物体追跡と骨格推定を統合した3次元空間表示を試みた(図\ref{fig:3d_multi}).2D-LiDAR点群に推定された骨格を重ねることで,カメラで見ているような直感的にわかりやすい計測が行えることが確認できた.一方で,曲線的な歩行や斜め方向の歩行に対しては,骨格が崩れる場面も見られた.

\begin{figure}[H]
     \centering
     \includegraphics[width=220pt]{ske_one_pre.png}
     \caption{直進歩行の骨格推定}
     \label{fig:ske}
\end{figure}



\begin{figure}[H]
     \centering
     \includegraphics[width=220pt]{ske_two.png}
     \caption{複数人の歩行における骨格推定と3次元描画}
     \label{fig:3d_multi}
\end{figure}

\section{おわりに}


本稿では,2D-LiDARによる足元計測データを利用して,歩行者追跡と三次元全身骨格推定を行うシステムを提案した.ByteTrackの導入により,従来のシステムよりも持続性の高い歩行者追跡を,リアルタイムで行うことが可能になった.また,検出・追跡と同時に全身骨格の推定が行えることも確認できた.

今後は,実環境における多様な歩行動作に対応するため,学習データの拡充などによる骨格推定精度の向上に取り組む.また,実環境での使用に向けて,より広範囲かつ多人数が混在する環境での性能について評価する必要がある.










\footnotesize
\bibliographystyle{junsrt}
\bibliography{books_pre}

\end{document}

