\documentclass[12pt,epsf,dvipdfmx]{jreport}
% \usepackage{graphicx} % 図の貼り付け用 (ics.styで読み込まれているためコメントアウト)
\usepackage{ics} % ICS卒論・修論スタイルファイル
\usepackage{makeidx} %索引生成用パッケージ
\usepackage{tabularx}% 横幅指定で表を作成
\usepackage{latexsym} % 数学記号用パッケージ

\usepackage{hhline}
\usepackage{here}

\usepackage{url}
\urlstyle{same}


\newtheorem{definition}{定義}[chapter]
\newtheorem{algorithm}{アルゴリズム}[chapter]

%% end of local definitions

\def\epsfsize#1#2{\ifnum#1>\hsize\hsize\else#1\fi}

\begin{document}

% \papercode{ICS-xxM-yyyyyyy} % 論文番号については、「論文番号および論文ファイル名について」のwebページを確認してください。
% \title{卒業論文、修士論文のタイトル}
% % \affiliation{工学部情報工学科}
% \affiliation{理工学研究科 数理電子情報専攻\\情報工学プログラム}
% % \affiliation{理工学研究科 数理電子情報コース}
% \author{氏 名}
% \date{令和7年2月10日提出}
% \supervisor{○○ ☆☆教授}
% \labname{○○研究室}
% \studentID{yyyyyyy}
% \maketitle

\papercode{ICS-24B-065}
\title{深層学習を用いた2D-LiDARaaaabbbbbによる歩行者追跡の高度化}
\affiliation{理工学研究科 数理電子情報専攻\\情報工学プログラム}
\author{廣中 優平}
\date{令和8年2月1日提出}
\supervisor{小林 貴訓教授}
\labname{小林研究室}
\studentID{24MM322}
\maketitle

\setcounter{page}{1}
\chapter*{概要}
 \pagenumbering{roman} % 消さない
 \addcontentsline{toc}{chapter}{概要} %消さない

人物の動きを計測することは,様々な場面で非常に重要である.近年では人と共存するロボットも多く登場し,人の位置を正しく検出・追跡したり,個人の振る舞いを理解することの重要性は一層増している.現在行われている人物計測の多くは,カメラ画像を用いて行われている.カメラ画像は情報量が多く,正確な計測を行うことができる.一方で,公共空間での撮影においては,不特定多数の人物の顔などが記録されるため,プライバシーの問題が発生する.

以前より我々の研究室では,足元の高さに設置した2D-LiDARを用いることで,前述の問題の解消を目指してきた\cite{hasan2021person}.水平面上の点群を高速に計測できる2D-LiDARを足元に設置することで,過分なプライバシー情報を取得することなく,歩行の様子を計測できる.これまでには,クラスタリングを用いた人物検出とカルマンフィルタによる追跡を用いて,精度の高い歩行者追跡が実現されてきた.一方で,複数人が歩行する場合に遮蔽が多発すると追跡性能が落ちる点,クラスタリングの処理速度が遅くリアルタイムでの追跡には向かない点などが問題点として挙げられていた.

また,人物の振る舞いを計測する手法として,2D-LiDARを使って歩行者の全身骨格を推定する研究も行われてきた\cite{s_b}.足元の高さに設置した2D-LiDARのみを用いて,人物の全身骨格を推定できる.2D-LiDARによって得られた歩行の様子を表す画像と,Kinectから取得した骨格情報を組み合わせて深層学習を行い,得られたモデルを使って推定を行う.足元情報のみから全身の骨格を推定することは,プライバシーに配慮した人物の振る舞いの計測につながる.

これらの技術はどれも非常に有用なものである.一方で,性能に不足がある点,リアルタイムで動作しない点,それぞれが個別に実装されている点など,実用に向けていくつかの解消すべき問題点がある.本研究では,歩行者追跡におけるいくつかの問題を解消すると共に,既存の骨格推定手法との融合を図っていく.2D-LiDARによって足元の様子を計測し,歩行者の検出と追跡を行う.更に,検出された人物の歩行の様子から三次元骨格を推定する.この一連の流れを同時に実行できる,実用性の高いシステムを目指した.本稿では,ByteTrackをベースとした2D-LiDARによる歩行者追跡の性能を定量的に評価し,従来手法との比較を行った.また,骨格推定手法と統合した統合システムを用いて複数人の歩行計測を行い,性能を確認した.


%M2中間のはじめに
% 人の動きを計測することは,様々な場面で非常に重要である.近年では公共空間で人と共存するロボットも多く登場し,人の位置を正しく検出・追跡したり,個人の振る舞いを理解することの重要性は一層増している.
% 現在行われている人物計測の多くは,カメラ画像を用いて行われている.カメラ画像は情報量が多く,正確な計測を行うことができる.
% 一方で,公共空間での撮影においては,不特定多数の人物の顔などが記録されるため,プライバシーの問題が発生する.
% 以前より我々の研究室では,足元の高さに設置した2D-LiDARを用いることで,前述の問題の解消を目指してきた\cite{hasan2021person}.
% 水平面上の点群を高速に計測できる2D-LiDARを足元に設置することで,過分なプライバシー情報を取得することなく,歩行の様子を計測できる.これまでには,クラスタリングを用いた人物検出とカルマンフィルタによる追跡を用いて,精度の高い歩行者追跡が実現されてきた.一方で,複数人が歩行する場合に遮蔽が多発すると追跡性能が落ちる点,クラスタリングの処理速度が遅くリアルタイムでの追跡には向かない点などが問題点として挙げられていた.
% また,人物の振る舞いを計測する手法として,2D-LiDARを使って歩行者の全身骨格を推定する研究も行われてきた\cite{suda_d}.足元の高さに設置した2D-LiDARのみを用いて,人物の全身骨格を推定できる.2D-LiDARによって得られた歩行の様子を表す画像と,Kinectから取得した骨格情報を組み合わせて深層学習を行い,得られたモデルを使って推定を行う.足元情報のみから全身の骨格を推定することは,プライバシーに配慮した人物の振る舞いの計測につながる.
% これらの技術はどれも非常に有用なものである.一方で,性能に不足がある点,リアルタイムで動作しない点,それぞれが個別に実装されている点など,実用に向けていくつかの解消すべき問題点がある.
% 本研究では,歩行者追跡におけるいくつかの問題を解消すると共に,既存の骨格推定手法との融合を図っていく.2D-LiDARによって足元の様子を計測し,歩行者の検出と追跡を行う.更に,検出された人物の歩行の様子から三次元骨格を推定する.この一連の流れを同時に実行できる,実用性の高いシステムを目指す.





%%% 謝辞 %%%

\chapter*{謝辞}
\addcontentsline{toc}{chapter}{謝辞} % 消さない

%卒論の謝辞
研究ならびに生活面においてご指導を賜りました小林教授,鈴木助教に深く感謝致します.

また,先輩としていつもよきアドバイスをくださり,快くデータを提供してくださりました須田悠介氏をはじめとする研究室の皆様,そして同期学生の皆様,並びに私を暖かく見守って頂いた両親はじめとする周囲のすべての皆様に深く感謝致します.

\tableofcontents % 目次生成
\listoffigures % 図目次生成
\addcontentsline{toc}{chapter}{図目次}
\listoftables % 表目次生成
\addcontentsline{toc}{chapter}{表目次}


\chapter{はじめに}
\pagenumbering{arabic} %消さないこと
\setcounter{page}{1}   %消さないこと

 \section{背景}
 背景


 \section{目的}
 目的


 \section{本論文の構成}
 構成



\chapter{実験}
 \section{実験方法}
 実験方法

 \section{実験結果}
 実験結果



\chapter{考察}
考察

\chapter{おわりに}
 \section{まとめ}
 まとめ

 \section{今後の課題}
 今後の課題


\chapter*{公表論文}
\addcontentsline{toc}{chapter}{公表論文}

\begin{list}%
 {} %default label
 {} %formatting parameter
 \item 査読付き論文
       \begin{itemize}
	\item Hogehoge: ....
       \end{itemize}
 \item 査読なし論文
       \begin{itemize}
	\item Hogehoge: ....
       \end{itemize}
\end{list}

\newpage

% 参考文献:bibtexを使う場合
%
\bibliographystyle{junsrt} % bstファイル名
\bibliography{books} % bibファイル名
\label{sannkoubunnkenn_chapter}

% 参考文献:直接記述する場合
% \begin{thebibliography}{99}
% \label{sannkoubunnkenn_chapter}
% 
% %% 程研究室参考文献形式
% 
% % 学術雑誌論文
% \bibitem{tagawa98}
% 多川 孝央, 大堀 順也, 程 京徳, 牛島 和夫: 相関論理における強相関性原理,
%         人工知能学会誌, Vol. 13, No. 3, pp. 387-394, 1998年5月.
% 
% \bibitem{Nonaka99}
% Yusuke NONAKA, Jingde CHENG, and Kazuo USHIJIMA: A Tasking Deadlock
%         Detector for Ada 95 Programs, Ada User Journal, Vol. 20, No. 1,
%         pp. 79-92, April 1999.
% 
% % 単行本
% \bibitem{ChengXX}
% 程 京徳: 相関論理入門, 何らか出版社, 200?年?月.
% 
% \bibitem{Jin01}
% Qun JIN, Jie LI, Nan ZHANG, Jingde CHENG, Clement YU, and Shoichi
%         NOGUCHI: Enabling Society with Information Technology,
%         Springer-Verlag, November 2001.
% 
% \bibitem{Hennessy93}
% Matthew Hennessy著, 荒木 啓二郎, 程 京徳 共訳:
% プログラミング言語の意味論入門, サイエンス社, 1993年12月.
% 
% % 単行本などに編集された章、論文
% \bibitem{Cheng91}
% Jingde CHENG: Relevance Logic and Entailment Logic, in I. Nakada and
%         M. Hagiya (Eds.), ``Software Science and Engineering,''
%         pp. 189-211, World Scientific, November 1991.
% 
% \bibitem{Nonaka00}
% Yusuke NONAKA, Jingde CHENG, and Kazuo USHIJIMA: A Supporting Tool for
%         Development of Self-measurement Ada Programs, in H. B. Keller
%         and E. Ploedereder (Eds.), ``Reliable Software Technologies -
%         Ada-Europe 2000, 5th International Conference on Reliable
%         Software Technologies, Potsdam, Germany, June 2000,
%         Proceedings,'' Lecture Notes in Computer Science, Vol. 1845,
%         pp. 69-81, Springer-Verlag, June 2000.
% 
% % 国際会議、国内シンポジウム論文集論文
% \bibitem{Goto01}
% Yuichi GOTO, Daisuke TAKAHASHI, and Jingde CHENG: Parallel Forward
%         Deduction Algorithms of General-Purpose Entailment Calculus on
%         Shared-Memory Parallel Computers, Proceedings of the ACIS 2nd
%         International Conference on Software Engineering, Artificial
%         Intelligence, Networking \& Parallel/Distributed Computing,
%         pp. 168-175, Nagoya, Japan, August 2001.
% 
% \bibitem{Koide01}
% 小出 雅人, 程 京徳: インターネット上でカードゲームを行うための汎用プロト
%         コル群の開発, 情報処理学会第6回ゲーム・プログラミング国際ワーク
%         ショップ論文集, pp. 78-85, 箱根, 日本, 2001 年10月.
% 
% % 博士,修士,卒業論文
% \bibitem{Goto05}
% 後藤 祐一: 強相関論理に基づいた自動前向き演繹とその応用, 埼玉大学大学院
% 	理工学研究科情報数理科学専攻博士論文,2005年3月.
% 
% \bibitem{Goto02}
% 後藤 祐一: 強相関論理と汎用前向き自動帰結演算システム EnCal を用いた知識
% 	発見, 埼玉大学大学院理工学研究科情報システム工学専攻修士論文,2003年
% 	2月.
% 
% \bibitem{Goto00}
% 後藤 祐一: 汎用前向き自動帰結演算システムEnCalの共有メモリ型並列計算機上
% 	での並列化, 埼玉大学工学部情報システム工学科卒業論文, 2001年2月.
% 
% 
% % 企業や製品のWebページなど
% % \newblockは後ろに続く文字列を一つの塊としてみなす命令
% \bibitem{cem}
% {Common Criteria Project}:
% \newblock CEM v3.1,
% \newblock \\http://www.commoncriteriaportal.org/thecc.html
% 
% \bibitem{ccportal}
% {Common Criteria Project}:
% \newblock Common Criteria Portal,
% \newblock \\http://www.commoncriteriaportal.org/
% 
% \end{thebibliography}

% 付録
\appendix % 以下,付録
\chapter{▽△▽△}
\section{ほげほげ}
\section{ほりゃほりゃ}
\chapter{▽△▽△}
\section{ほげほげ}
\section{ほりゃほりゃ}

% 索引. mendexかmkindexで作成
\printindex

%目次にIndexを表示させる場合はmendexかmkindex実行後に作成される
% hogehoge.idmファイルのtheindex環境の中で下記命令を置く
%\addcontentsline{toc}{chapter}{Index}

\end{document}

